\section{Prerequisites}
\label{sec:prerequisites}

  In this section, we will give a brief overview of the terms and definitions
  that we will use throughout this report. These definitions should not be
  considered complete or even absolutely accurate, as they are a personal
  choice of words with the goal of giving the reader an impression of what
  we are talking about.

  \subsection{Neural networks}
  \label{sub:neural_networks}
  
    A \textit{Neuron} is a programmatic structure (or function) that is based
    on neurons inside the brain. In its most basic form, it takes a vector of
    input variables, calculates a weighted sum and applies an activation function
    (typically a sigmoid function like \textit{tanh}).

    A \textit{neural network} is a collection of neurons which are typically
    arranged in layers, where the input of a layer is the output of the
    preceeding one. The output of the last layer is called the output of
    the network.

  \subsection{Recurrent Neural Networks}
  \label{sub:recurrent_neural_networks}
  
    A \textit{recurrent neural network} is a neural network where the
    output of one or more layers is fed back as input to the network
    or a selection of units. This allows the network to make decisions
    based on previous results / events.

  \subsection{LSTM and GRU}
  \label{sub:lstm_and_gru}
  
    TODO

  \subsection{Embeddings}
  \label{sub:embeddings}
  
    TODO

  \subsection{Keras}
  \label{sub:keras}
  
    Keras is a Python module that contains functions and classes to easily create
    and manipulate neural networks. It contains not only basic neurons, but
    also more involved units like LSTM-Cells or Embedding-Layers (plus
    many more).
