\section{Introduction}
\label{sec:introduction}

  Machine learning has recently become more popular than ever before and
  is being applied in research disciplines far beyond computer science
  (for example in the humanities to analyze large quantities of ancient
  texts). This has resulted in the necessaty for many students to learn
  how to program. Being pressured with assignments and deadlines, students
  tend to be satisfied with programs that 'get the work done', but often
  suffer from low quality. It is common to avoid error checking, copy
  and paste code from the internet or stick to very low level language
  constructs instead of utilizing more advanced techniques that get the
  job done more efficiently and with actually less hassle for the one
  that is writing it (think of list comprehensions in Python for example).

  Becoming a good programmer is a process that takes a lot of time
  and devotion and ideally a mentor that can help get things done 'the
  right way', but often students (especially in the humanities) may be
  devoted enough, but lack the time and a mentor to become proficient,
  which is very unfortunate.

  Our goal is to provide such a mentor for aspiring programmers in the
  form of a program that utilizes artificial intelligence to find best
  practices employed in a programming language and make suggestions
  based on its own findings. In theory it should be possible to apply
  machine learning techniques on a big collection of source code that
  is considered high quality and find patterns, just as companies like
  Netflix or Amazon find patterns in their customers preferences. In this
  paper, we will not just report our findings, but will also emphasize
  the steps that had to be taken in order to get there.
