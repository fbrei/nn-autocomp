\section{Data preparation} \label{sec:data_preparation}

  Data preparation is one of the most important aspects in machine
  learning.  While it is nice to have huge amounts of data at your
  disposal, there is also a high chance that it contains a lot of noise.

  \subsection{Dealing with comments} \label{sub:dealing_with_comments}

    One kind of noise is introduced by comments within the source
    code. Although these may be informative to the reader, they will
    probably just distract our algorithm from learning the things that
    it should.

    We addressed the issue about the comments with a simple substitution
    using regular expressions, after reading every single source file
    into a single string variable called \verb+blob+. The relevant part
    of our source code can be seen in (\ref{fig:comments}).

    \begin{figure}[htpb]
      \centering \begin{lstlisting}[language=Python]
        import re

        comments = re.compile(r'#.*')
        cleaned_text = comments.sub('', blob)
      \end{lstlisting} \caption{Removing all comments in Python source
      code} \label{fig:comments}
    \end{figure}

    This may seem specific to Python, but it is easy to add another line
    that deletes every substring starting with something like \verb+//+
    or that is wrapped between \verb+/*...*/+, covering a great deal of
    programming languages.

  \subsection{Punctuation and Operators}
  \label{sub:punctuation_and_operators}

    Another potential source of trouble is the fact that it is
    absolutely common to append something like a dot or comma directly
    to a name without adding whitespace in between. While perfectly
    clear to humans, a machine does not recognize \verb+a,+ and
    \verb+a+\textvisiblespace\verb+,+ as the same thing.

    We can easily remedy this by adding said whitespace, again using
    regular expressions.  Example (\ref{fig:puncts}) deals with
    dots and commas, but we have also used this on the characters
    \verb+(,),[,],{,}+ and \verb+=+.

    \begin{figure}[htpb]
      \centering \begin{lstlisting}[language=Python]
        commas = re.compile(r',')
        cleaned_text = commas.sub(' , ', cleaned_text)

        dots = re.compile(r'\.') 
        cleaned_text = dots.sub(' . ', cleaned_text)
      \end{lstlisting} \caption{Seperating punctuation} \label{fig:puncts}
    \end{figure}

    This obviously introduces a new problem as operators like \verb+==+
    or \verb++=* are torn apart.
