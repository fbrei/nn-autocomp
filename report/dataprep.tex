\section{Data preparation} \label{sec:data_preparation}

  Data preparation is one of the most important aspects in machine
  learning.  While it is nice to have huge amounts of data at your
  disposal, there is also a high chance that it contains a lot of noise.

  \subsection{Dealing with comments} \label{sub:dealing_with_comments}

    One kind of noise is introduced by comments within the source
    code. Although these may be informative to the reader, they will
    probably just distract our algorithm from learning the things that
    it should.

    We addressed the issue about the comments with a simple substitution
    using regular expressions, after reading every single source file
    into a single string variable called \verb+blob+. The relevant part
    of our source code can be seen in (\ref{fig:comments}).

    \begin{figure}[htpb]
      \centering \begin{lstlisting}[language=Python]
        import re

        comments = re.compile(r'#.*')
        cleaned_text = comments.sub('', blob) \end{lstlisting} 
        \caption{Removing all comments in Python source
      code} \label{fig:comments}
    \end{figure}

    This may seem specific to Python, but it is easy to add another line
    that deletes every substring starting with something like \verb+//+
    or that is wrapped between \verb+/*...*/+, covering a great deal of
    programming languages.

  \subsection{Punctuation and Operators}
  \label{sub:punctuation_and_operators}

    Another potential source of trouble is the fact that it is
    absolutely common to append something like a dot or comma directly
    to a name without adding whitespace in between. While perfectly
    clear to humans, a machine does not recognize \verb+a,+ and
    \verb+a+\textvisiblespace\verb+,+ as the same thing.

    We can easily remedy this by adding said whitespace, again using
    regular expressions.  Example (\ref{fig:puncts}) deals with
    dots and commas, but we have also used this on the characters
    \verb+(,),[,],{,}+ and \verb+=+.

    \begin{figure}[htpb]
      \centering \begin{lstlisting}[language=Python]
        commas = re.compile(r',')
        cleaned_text = commas.sub(' , ', cleaned_text)

        dots = re.compile(r'\.') 
        cleaned_text = dots.sub(' . ', cleaned_text) \end{lstlisting} 
        \caption{Seperating punctuation} 
        \label{fig:puncts}
    \end{figure}

    This obviously introduces a new problem as operators like \verb+==+
    or \verb|+=| are torn apart. This will again be fixed using regular expressions,
    as seen in (\ref{fig:operators}).

    \begin{figure}[htpb]
      \centering
      \begin{lstlisting}[language=Python]
        cleaned_text = re.sub(r'=  =', '==', cleaned_text)                                                                │                                                                                                        
        cleaned_text = re.sub(r'\+ =', '\+=', cleaned_text) \end{lstlisting}
      \caption{Fixing operators that got seperated}
      \label{fig:operators}
    \end{figure}

  \subsection{Turning words into numbers}
  \label{sub:turning_words_into_numbers}
  
    Since neural networks can only deal with numeric values, we have to turn all the words into numbers.
    This can easily be achieved by assigning sequential numbers to all the words, as seen in (\ref{fig:tonums}).
    At first, we make sure that there are no words consisting only of whitespace. Then we add a special token
    that will be used later in all places that we think are variable names, constants or other kinds
    of identifiers. After that we make all the items so far unique. If this was omitted, the length of the dictionary
    and the highest index number would no longer correspondend, which would cause a lot of overhead when we
    create a one hot encoding later. Then we use dictionary comprehension to create the dictionaries.

    It is beneficial for the performance to create dictionaries for both directions of the lookup.

    \begin{figure}[htpb]
      \centering
      \begin{lstlisting}
  cleaned_word_list = [ w.strip() for w in word_list if w.strip() != '' ]
  cleaned_word_list += ['<ID>']
  cleaned_word_list = set(cleaned_word_list)
  word_idx_pairs = list(enumerate(cleaned_word_list))

  w2n_dict = { w: i for i, w in word_idx_pairs}
  n2w_dict = { i: w for i, w in word_idx_pairs} \end{lstlisting}
      \caption{Turning words into numerical values}
      \label{fig:tonums}
    \end{figure}

    Just like any other language, a programming language follows the rule that there is a small number of words that
    appear extremely often, while the majority of words appears far less frequently. We use this to
    find out which words are keywords (like \verb+if+) or just common enough to be considered part of the language
    (like the string \verb+'__name__'+) and which words are just noise. The tokens can then mentally be split into
    two categories, those that carry meaning to the language and those that are just identifiers and will all be treated
    as equal later on.
  
    So after splitting the whole collection of source code based on whitespace, we decided that a word needs the appear
    in at least a certain percentage of all documents to be considered meaningful (\ref{fig:downsamp}).

    \begin{figure}[htpb]
      \centering
  \begin{lstlisting}
from collections import Counter
all_words = cleaned_text.split(" ")
c = Counter(all_words)

min_fraction = 0.2
min_count = num_files * min_fraction

reduced_words = [ w for w in all_words if c[w] >= min_count ] \end{lstlisting}
      \caption{Cutting off uncommon tokens}
      \label{fig:downsamp}
    \end{figure}

    Of course \verb+min_fraction+ is another parameter that can be tuned. We experimented with numbers 
    from $0.01$ up to and including $0.2$ and received equally good results. We decided to go with $0.2$ because this
    reduces the dictionary size from $151441$ to $214$ and thereby greatly reducing the amount of RAM needed
    to store the training data. Plus, if there are too many tokens in the dictionary the network might
    try to learn to differentiate between objects that actually mean the same.

    Finally, the sequences of words can be turned into sequences of numbers, keeping all the ones that appear in
    our dictionary natively and replacing everything else with the special token \verb+<ID>+. Translating
    these numbers back into words than gives us lines like (\ref{fig:goodsample})

    \begin{figure}[htpb]
      \centering
      \begin{verbatim}
    ['<ID>', 'def', '<ID>', '(', '<ID>', ')', ':', '<ID>'] \end{verbatim}
      \caption{Sample line after translating the words into numbers}
      \label{fig:goodsample}
    \end{figure}

    Apart from the \verb+<ID>+ tokens at the beginning and end, this is exactly what we are looking for. A scriptable text editor
    can use this type of output to make smart suggestions and just ask the programmer to fill out the \verb+<ID>+ placholders.

    \subsection{Creating the training data}
    \label{sub:creating_the_training_data}
    
      Creating samples for training from these sequences is now rather trivial. Given an input length \verb+w+, we need
      to create input sequences from a starting index \verb+i+ up to \verb|i+w-1| and define the word at position
      \verb|i+w| as the target value (\ref{fig:tosamples}). Since we replaced the line breaks in our input with
      \verb+<EOL>+ tags and split the text based on these, we have to reintroduce them to make sure that
      the network can also learn when to break a line.

      \begin{figure}[htpb]
        \centering
        \begin{lstlisting}
    ls = sentence + [ reduced_w2n_dict['<EOL>'] ]
    
    x_vals = [ ls[i:i+window_size] for i in range(len(ls) - window_size) ]
    y_vals = [ ls[i+window_size] for i in range(len(ls) - window_size) ] \end{lstlisting}
        \caption{Turning the sequences into input label pairs}
        \label{fig:tosamples}
      \end{figure}

      After that, the input data is turned into a numpy-array, while the labels are turned into one hot encoded vectors.
      This is crucial in classification tasks, as a small deviation around the correct target value could lead to
      wrong predictions.
